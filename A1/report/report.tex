% !TeX root = ./report.tex
\documentclass[12pt]{article}

\usepackage{graphicx}
\usepackage{paralist}
\usepackage{listings}
\usepackage{booktabs}
\usepackage{hyperref}
\hypersetup{
    colorlinks=true,
    linkcolor=blue,
    filecolor=magenta,      
    urlcolor=cyan,
}


\oddsidemargin 0mm
\evensidemargin 0mm
\textwidth 160mm
\textheight 200mm

\pagestyle {plain}
\pagenumbering{arabic}

\newcounter{stepnum}

\title{Assignment 1 Solution}
\author{Jay Mody}
\date{\today}

\begin {document}

\maketitle

This report outlines the results of implementing and testing two modules in python: DateT, an ADT that represents date, and GPosT, and ADT that represents position. These modules are implemented using a given design specification. This report will also discuss critiques of the given design specifications, and answer questions about software practice and engineering as a discipline in general.

\section{Testing of the Original Program}
\subsection{Assumptions}
\subsubsection{DateT ADT}
I based my assumptions for the DateT ADT off of the python datetime module implementation (taken from \url{docs.python.org}), namely:
\begin{quotation}
  "An idealized naive date, assuming the current Gregorian calendar always was, and always will be, in effect. "
\end{quotation}
This includes:
\begin{itemize}
  \item The calendar has three main attributes, a year, month, and day.
  \item The first year is 1, and the last year is 9999.
  \item A year contains 12 months, with each month containing the following number of days (in order):
  \begin{itemize}
      \item January (31 days)
      \item February (28 days, 29 on a leap year)
      \item March (31 days)
      \item April (30 days)
      \item May (31 days)
      \item June (30 days)
      \item July (31 days)
      \item August (31 days)
      \item September (30 days)
      \item October (31 days)
      \item November (30 days)
      \item December (31 days)
  \end{itemize}
  \item As a result of the above, a year contains 365 days, except on leap years, where there is an additional day in February, making a leap year contain 366 days.
  \item Leap years happen every 4 years, starting from year 4. Leap years do not occur on years that are a multiple of 100, unless they are also a multiple of 400 (ie 300 is not a leap year but 800 is).
\end{itemize}

\subsubsection{GPosT ADT}
I based my assumptions for the GPosT ADT off of the website \url{https://www.movable-type.co.uk/scripts/latlong.html}. Namely:
\begin{itemize}
  \item The longitude and latitude are represented as signed decimal degrees, where longitude (represented by the symbol $\lambda$) must be on the range [-180, 180], and latitude (represented by the symbol $\phi$) must be on the range [-90, 90].
  \item The distance and move functions are modeled by the equations provided by the website (with the distance function specifically using the Haversine formula).
  \item Speed and distance are measured in terms of km and hours, and may be negative (indicates opposite direction) with no restriction on the input range.
  \item Bearing also has no restriction on the input range, and is represented as a signed decimal degree.
\end{itemize} 

\subsection{Approach}
My test approach involved creating 2-4 test cases for each function. I tried to include the following types of test cases for each function:
\begin{itemize}
  \item A trivial "normal" test case
  \item A trivial edge case (-1, 0, max limit, boundary testing, month changes, etc ...)
  \item A non-trivial edge case (leap year)
\end{itemize}

Additionally, for the constructors, I tested a range of both valid and invalid inputs to make sure the correct errors were being raised for invalid inputs, and aren't being raised for valid ones.

\subsection{Results}
Below I put the log of the pytest results after running the test driver on my code (spoiler alert, I passed them all):

\begin{lstlisting}
  =========================================================================================== test session starts ============================================================================================
  platform darwin -- Python 3.7.5, pytest-5.3.4, py-1.8.1, pluggy-0.13.1 -- /Users/jay/.pyenv/versions/3.7.5/envs/base/bin/python
  cachedir: .pytest_cache
  rootdir: /Users/jay/code/edu/2aa4/A1
  plugins: dash-1.7.0
  collected 20 items
  
  src/test_driver.py::test_DateT_init PASSED
  src/test_driver.py::test_DateT_day PASSED
  src/test_driver.py::test_DateT_month PASSED
  src/test_driver.py::test_DateT_year PASSED
  src/test_driver.py::test_DateT_equal PASSED
  src/test_driver.py::test_DateT_next PASSED
  src/test_driver.py::test_DateT_prev PASSED
  src/test_driver.py::test_DateT_before PASSED
  src/test_driver.py::test_DateT_after PASSED
  src/test_driver.py::test_DateT_add_days PASSED
  src/test_driver.py::test_DateT_days_between PASSED
  src/test_driver.py::test_GPosT_init PASSED
  src/test_driver.py::test_GPosT_lat PASSED
  src/test_driver.py::test_GPosT_long PASSED
  src/test_driver.py::test_GPosT_west_of PASSED
  src/test_driver.py::test_GPosT_north_of PASSED
  src/test_driver.py::test_GPosT_distance PASSED
  src/test_driver.py::test_GPosT_equal PASSED
  src/test_driver.py::test_GPosT_move PASSED
  src/test_driver.py::test_GPosT_arrival_date PASSED
  
  ============================================================================================ 20 passed in 0.09s ============================================================================================
\end{lstlisting}

\section{Results of Testing Partner's Code}

Consequences of running partner's code.  Success, or lack of success, running
test cases.  Explanation of why it worked, or didn't.

\section{Critique of Given Design Specification}

Advantages and disadvantages of the given design specification.

\section{Answers to Questions}

\begin{enumerate}[(a)]

\item 

\end{enumerate}

\newpage

\lstset{language=Python, basicstyle=\tiny, breaklines=true, showspaces=false,
  showstringspaces=false, breakatwhitespace=true}
%\lstset{language=C,linewidth=.94\textwidth,xleftmargin=1.1cm}

\def\thesection{\Alph{section}}

\section{Code for date\_adt.py}

\noindent \lstinputlisting{../src/date_adt.py}

\newpage

\section{Code for pos\_adt.py}

\noindent \lstinputlisting{../src/pos_adt.py}

\newpage

\section{Code for test\_driver.py}

\noindent \lstinputlisting{../src/test_driver.py}

\newpage

\section{Code for Partner's CalcModule.py}

\noindent \lstinputlisting{../partner/pos_adt.py}

\end {document}